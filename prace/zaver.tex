\chapter*{Závěr}
\addcontentsline{toc}{chapter}{Závěr}
Pro dvě molekuly, BeH a OH, jsme získali křivky excitovaných stavů molekuly pro použití v 
R-maticových výpočtech i referenční křivky pro nastavení parametrů těchto výpočtů.
Pro molekulu BeH jsme při referenčních výpočtech metodou MRCI získali křivky dostatečně 
přesné pro nastavení R-maticových výpočtů. Pro popis excitovaných stavů neutrální 
molekuly BeH pro R-maticové výpočty pak navrhujeme použít metodu CAS-SCF s aktivním 
prostorem 6,2,2,0 v aug-cc-pVDZ bázi. Pro popis excitovaných stavů molekuly BeH je 
dle našich zjištění nutné používat augmented báze.

Pro molekulu OH jsme metodou MRCI v aug-cc-pV5Z získali křivky konzistentní s
experimentálními daty, které lze použít pro nastavení R-maticových výpočtů.
Zároveň jsme nalezli vhodný popis excitovaných stavů pomocí metod CAS-SCF s aktivními
prostory 6,2,2,0 a 7,3,3,0 v aug-cc-pVDZ, spolu s vhodným nalezením vah jednotlivých 
stavů vylepšující konverenci stavů, které navrhujeme pro použití v R-maticových 
výpočtech. Augmentované báze se ukázaly být lepší v popisu excitovaných stavů této 
molekuly, ačkoliv rozdíl nebyl tak zásadní jako u molekuly BeH.

Na těchto výsledcích budou dále prováděny R-maticové výpočty, které nebyly z časových důvodů zahrnuty již do této práce.