Pro provedení v R-maticových výpočtech je třeba najít dobrý popis excitovaných stavů 
neutrální molekuly v okolí rovnovážné geometrie, 
provedené ab-initio metodou v relativně malé bázi. Dále  pro nastavení některých 
parametrů těchto výpočtů je třeba mít referenční 
potenciálové křivky neutrální molekuly i aniontu, které jsou konzistentní s 
experimentálně získanými hodnotami. V této práci se 
zabýváme nalezením popisu excitovaných stavů i získáním referenčních křivek za účelem 
provedení R-maticových výpočtů pro molekuly 
BeH a OH. 
Pro obě zmíněné molekuly jsme nalezli referenční křivky použitelné v těchto výpočtech. 
Pro BeH navrhujeme popis metodou SA-CASSCF v 
aktivním prostoru 6,2,2,0 a aug-cc-pVDZ bázi, pro OH pak SA-CASSCF v aktivním prostoru 
6,2,2,0 nebo 7,3,3,0 a aug-cc-pVDZ  bázi, kdy 
jsme navíc nalezli nastavení vah jednotlivých stavů výrazně zlepšující tvar křivek.
R-maticoové výpočty těchto datech se nám na z časových důvodů zatím nepovedlo provést.
