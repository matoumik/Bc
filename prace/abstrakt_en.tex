For succesfully carrying out R-matrix calculations, a good description of the excited states of the neutral molecule around the equilibrium is needed, obtained by an ab-initio method in a relatively small basis. It is also neccesary to have potential curves of the neutral molecule and the anion that are consistent with the experimentally obtained values for the molecule and are used to set up the initial parameters of these calculations.
In this work
we are trying to find a description of the excited states and obtain reference curves in order to perform R-matrix calculations for two molecules, BeH and OH.
For both of these molecules, we found the reference curves with enough accuracy to be 
used in R-matrix calculations. For BeH we propose a description of the excited states by 
the SA-CASSCF method with an active space of 6,2,2,0 and in the aug-cc-pVDZ basis. 
Similarly for OH a description by the SA-CASSCF method with an active space of 6,2,2,0 or 
7,3,3,0 and an in an aug-cc-pVDZ basis should be used,
where we have also found a setting of the weights of the states in the SA-CAS-SCF method 
significantly improving the shape of the curves.
We have not yet been able to perform the R-matrix calculations because of insufficient 
time.
