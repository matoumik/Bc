\chapter*{Úvod}
\addcontentsline{toc}{chapter}{Úvod}
Při srážce volného elektronu s dvouatomovou molekulou může dojít k rezonančnímu
zachycení elektronu a přechodnému vzniku aniontu této molekuly. Ten pak následně může 
zanikat různými cestami, například se může elektron znovu odtrhnout, přičemž zanechá 
molekulu v excitovaném stavu,
případně může dojít k zachycení elektronu jedním z atomů a disociaci molekuly, nazývaný 
disociační záchyt.
Lze studovat i opačný proces, asociativní odtržení, kdy kombinací dvou atomů, z nichž 
některý nese záporný náboj, vznikne neutrální molekula a volný elektron.

K popisu zmíněných jevů je třeba znát potenciálové křivky neutrální molekuly i aniontu.
Pro neutrální molekulu lze použít křivky získat kvantově chemickými výpočty. Pro anion 
lze kvantově chemické výpočty na určitém úseku potenciálové křivky použít, jen pokud 
energie křivky v dané oblasti je nižší než energie základního stavu neutrální molekuly. 
V opačném případě není nadbytečný elektron vázaný, ale přejde v rezonanci, což vede k 
selhání kvantově chemických metod.

V takových oblastech je třeba získat výsledky, které pak použijeme v jaderné dynamice, 
pomocí kvantové teorie rozptylu s použitím aproximace pevných jader.
Jedna z metod vhodných pro tento účel je výpočet pomocí R-matic, což je metoda původně 
vyvinutá pro použití v jaderné fyzice, ale ukázala se být užitečná i v rozptylu na 
atomech a molekulách.
Ta rozděluje prostor,na němž probíhá výpočet na dvě oblasti, vnitřní, kde se řeší celý 
problém s okrajovými podmínkami, a vnější, kde se uvažuje jen 
vlnová funkce volného elektronu, a v této oblasti se hledají taková řešení, která na 
rozhraní s vnitřní oblastí navazují.
V R-maticových výpočtech se používá báze popisující kontinuum, která se ve vnitřní  
oblasti rozšíří o vlnové funkce popisující stavy neutrální molekuly.

V R-maticových výpočtech se používají některé výsledky kvantové chemie. Pro 
výpočty je třeba najít metodu a bázi dobře popisující základní i excitované stavy 
neutrální molekuly, 
v R-maticové teorii i dále v této práci obvykle nazývané target.
Tato metoda musí být provedena v relativně malé bázi a musí dávat dostatečně přesné 
výsledky pro geometrie na kterých provádíme rozptylový výpočet, v našem případě se 
jedná o okolí rovnovážné vzdálenosti základního stavu neutrální molekuly.
Dále je nutné nalézt potenciálové křivky základního stavu neutrální molekuly i aniontu, 
které jsou dostatečně přesné, aby je bylo možné použít jako referenční data pro 
nalezení vhodného nastavení vstupních parametrů u R-maticových výpočtů.

Tato práce se bude zabývat hledáním a srovnáním 
různých popisů na dvou konkrétních molekulách, BeH a OH,
pro následné provedení R-maticových výpočtů.

