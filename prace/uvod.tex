\chapter*{Úvod}
\addcontentsline{toc}{chapter}{Úvod}

Jedna z numerických metod v kvantové teorii rozptylu je výpočet pomocí R-matic.

Ta je podobná metodám kvantové chemie a některé z nich př\TD

Pro R-maticové výpočty jsou důležité dva typy výpočtů. Primárně referenční výpočty, 
kterými je třeba 
získat disociační křivku jak neutrální molekuly, tak aniontu, co nejpřesnější tvarem i 
relativní pozicí vůči sobě, které se pak používají pro kalibraci R-maticových výpočtů.
Dále je třeba najít dobrý popis targetu, tedy metodu, která dobře popisuje i základní i 
excitované stavy dané molekuly. Ta se pak použije v R-maticových výpočtech, 
s bází rozšířenou o funkce popisující pohyb volného elektronu. Tyto tyto výpočty jsou výrazně náročnější, a proto je třeba, aby tato metoda byla sama nepříliš náročná a v relativně malé bázi dávala rozumně přesné výsledky v okolí rovnovážné geometrie.