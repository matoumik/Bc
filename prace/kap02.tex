\chapter{Výsledky}
Zkoumali jsme molekuly $\mathrm{BeH/BeH^-}$ a $\mathrm{OH/OH^-}$, protože se jedná o 
molekuly, které mají vázaný jak základní stav, tak anion, a zároveň se jedná o 
dostatečně malé systémy, aby bylo možné provádět výpočty přesnými metodami.

Ke kvantově chemickým výpočtům jsme používali program MOLPRO.\cite{MOLPRO-WIREs}\cite{MOLPRO}. 
Pro určitý soubor mezijaderných vzdáleností\footnote{cca 35 hodnot} jsme napočítali 
energii základního stavu molekuly pro fixovaná jádra, u základního stavu, tak u 
aniontu. Ze znalosti těchto křivek jsme poté zjišťovali parametry molekul, které je 
možné nalézt experimentálně, což jsou disociační energie aniontu i neutrální molekuly a 
elektronové afinity vázané i úplně disociované\footnote{Ta odpovídá odpovídá 
elektronové afinitě některého z prvků v molekule.} molekuly. Protože ale experimentální 
data nejsou určená minimem potenciální energie molekuly, ale základní vibrační 
hladinou, bylo třeba napočítat energetické hladiny získaného potenciálu. K tomu jsme 
použili \? metodu výpočtu na na mříži \? . Vzhledem k počtu geometrií, pro které jsme 
prováděli kvantově-chemické výpočty, který byl nedostatečný pro další numerické výpočty
\footnote{a extrémní výpočetní náročnosti při případném výpočtu v dostatečném počtu 
geometrií}, jsme získané hodnoty proložili kubickým splinem, ze kterého jsme pak 
interpolovali hodnotu potenciálu v několika stovkách bodů. Poté jsme numericky získali 
energetické hladiny v daném potenciálové křivce pro částici o (redukované) hmotnosti $
\mu = m_1m_2/(m_1+m_2)$, kde $m_1,m_2$ jsou hmotnosti jednotlivých jader. Získané 
vlastní stavy pak odpovídají vibračním stavům dané molekuly.

\begin{figure}
\centering
\includegraphics[width=0.8\textwidth]{../img/VibrStavy1.png}
\caption{Nejnižší vibrační hladiny molekul $\mathrm{BeH/BeH^-}$}
\label{Vibr1}
\end{figure}
\ltable{../tbl/BeH.tex}{BeH}
\ltable{../tbl/OH.tex}{OH}


\begin{tabular}{lllll}
\toprule
Method & $v_0 [\mathrm{eV}]$ & $v_1 [\mathrm{eV}]$ & $v_2 [\mathrm{eV}]$ & $v_3[\mathrm{eV}]$ \\ \midrule
    Exper. & 0.0000 & 0.2463 & 0.4831 & 0.7103 \\
\midrule
FCI/aug-cc-pVDZ & 0.0000 & 0.2418 & 0.4740 & 0.6962\\
FCI/aug-cc-pCVDZ & 0.0000 & 0.2423 & 0.4748 & 0.6974\\
FCI/cc-pVTZ & 0.0000 & 0.2444 & 0.4795 & 0.7050\\
FCI/aug-cc-pVTZ & 0.0000 & 0.2439 & 0.4785 & 0.7034\\
RCCSD(T) /aug-cc-pVTZ & 0.0000 & 0.2447 & 0.4801 & 0.7060\\
MRCI 5,1,1,0/aug-cc-pVTZ & 0.0000 & 0.2440 & 0.4785 & 0.7034\\
MRCI 6,2,2,0/aug-cc-pVTZ & 0.0000 & 0.2439 & 0.4785 & 0.7034\\
MRCI 6,2,2,0/aug-cc-pVTZ & 0.0000 & 0.2439 & 0.4785 & 0.7034\\
MRCI 5,1,1,0/aug-cc-pVQZ & 0.0000 & 0.2456 & 0.4817 & 0.7081\\
MRCI 6,2,2,0/aug-cc-pVQZ & 0.0000 & 0.2455 & 0.4816 & 0.7080\\
MRCI 9,3,3,1/aug-cc-pVQZ & 0.0000 & 0.2455 & 0.4816 & 0.7080\\
MRCI 6,2,2,0/aug-cc-pV5Z & 0.0000 & 0.2458 & 0.4820 & 0.7086\\
MRCI 6,2,2,0/CBE 5,Q & 0.0000 & 0.2459 & 0.4824 & 0.7094\\
MRCI 6,2,2,0/CBE Q,T & 0.0000 & 0.2461 & 0.4828 & 0.7097\\
FCI/CBE T,D & 0.0000 & 0.2447 & 0.4801 & 0.7058\\
RCCSD(T) /aug-cc-pVQZ & 0.0000 & 0.2462 & 0.4832 & 0.7105\\
\bottomrule
\end{tabular}

\begin{tabular}{lllll}
\toprule
Method & $v_0 [\mathrm{eV}]$ & $v_1 [\mathrm{eV}]$ & $v_2 [\mathrm{eV}]$ & $v_3[\mathrm{eV}]$ \\ \midrule
Exper. & 0.000 & 0.4437 & 0.8667 & 1.2696 \\
\midrule

MRCI 8,2,2,0/aug-cc-pVDZ & 0.0000 & 0.4351 & 0.8489 & 1.2418\\
MRCI 4,1,1,0/aug-cc-pVTZ & 0.0000 & 0.4421 & 0.8636 & 1.2646\\
MRCI 4,2,2,0/aug-cc-pVTZ & 0.0000 & 0.4408 & 0.8608 & 1.2593\\
MRCI 6,2,2,0/aug-cc-pVTZ & 0.0000 & 0.4415 & 0.8627 & 1.2638\\
MRCI 8,2,2,0/aug-cc-pVTZ & 0.0000 & 0.4410 & 0.8616 & 1.2620\\
MRCI 6,2,2,0/aug-cc-pVQZ & 0.0000 & 0.4437 & 0.8669 & 1.2698\\
MRCI 8,2,2,0/aug-cc-pVQZ & 0.0000 & 0.4431 & 0.8657 & 1.2679\\
MRCI 6,2,2,0/aug-cc-pV5Z & 0.0000 & 0.4442 & 0.8678 & 1.2712\\
MRCI 8,2,2,0/aug-cc-pV5Z & 0.0000 & 0.4436 & 0.8666 & 1.2693\\
MRCI 8,2,2,0/CBE 5,Q & 0.0000 & 0.4422 & 0.8617 & 1.2587\\
MRCI 6,2,2,0/CBE 5,Q & 0.0000 & 0.4445 & 0.8685 & 1.2723\\
\bottomrule
\end{tabular}

\begin{table}[]
\centering
\caption{OH- vibration states}
\label{TODO}
\begin{tabular}{rrrrr}
\toprule
Method & $v_0 [" eV"]$ & $v_1 [" eV"]$ & $v_2 [" eV"]$ & $v_3[" eV"]$ \\ \midrule
CI 4,1,1,0 /aug-cc-pVDZ & 0.228 & 0.662 & 1.069 & 1.449\\
CI 4,2,2,0 /aug-cc-pVDZ & 0.221 & 0.647 & 1.047 & 1.426\\
CI 6,2,2,0 /aug-cc-pVDZ & 0.224 & 0.654 & 1.059 & 1.440\\
CI 8,2,2,0 /aug-cc-pVDZ & 0.224 & 0.653 & 1.057 & 1.437\\
CI 4,1,1,0 /aug-cc-pVTZ & 0.212 & 0.636 & 1.047 & 1.448\\
CI 4,2,2,0 /aug-cc-pVTZ & 0.221 & 0.649 & 1.054 & 1.439\\
CI 6,2,2,0 /aug-cc-pVTZ & 0.225 & 0.661 & 1.072 & 1.460\\
CI 8,2,2,0 /aug-cc-pVTZ & 0.225 & 0.660 & 1.070 & 1.457\\
CI 4,1,1,0 /aug-cc-pVQZ & 0.212 & 0.637 & 1.051 & 1.455\\
CI 4,2,2,0 /aug-cc-pVQZ & 0.224 & 0.657 & 1.065 & 1.450\\
CI 6,2,2,0 /aug-cc-pVQZ & 0.225 & 0.663 & 1.077 & 1.467\\
CI 8,2,2,0 /aug-cc-pVQZ & 0.225 & 0.662 & 1.075 & 1.463\\
\bottomrule
\end{tabular}
\end{table}
    