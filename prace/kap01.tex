\chapter{Teorie}
\section{}
Základní cíl kvantově chemických výpočtů je najít řešení stacionární schrödingerovy rovnice
\begin{equation}
\op{H}\ket{\Psi} = E\ket{\Psi},
\end{equation}
kde $\ket{\Psi}$ je mnohačásticová vlnová funkce a $\op{H}$ je hamiltonián popisující daný 
systém. Protože se jedná o dost složitý problém na numerické výpočty, je nutné zavést 
několik zjednodušení. První je Born-Oppenhaimerova aproximace, která vzhledem k řádově 
rozdílným hmotnostem jader a elektronů rozděluje pohyb jader a elektronů, čímž pádem 
$\ket{\Psi}$ závisí na souřadnicích jader jen parametricky, 
a nevystupují jako proměnné v 
řešené rovnici. Další je popis mnohaelektronové vlnové funkce
$\ket{\Psi}:= f(\vec{x_1} \dots \vec{x_i} \dots \vec{x_N})$, kde $\vec{x_i}$ jsou 
polohy jednotlivých elektronů, pomocí součinu jednoelektronových funkcí
$\ket{\Psi}:= f_1(\vec{x_1})f_2(\vec{x_2})\dots f_i(\vec{x_i})\dots f_N(\vec{x_N})$.
Toto ale naráží na problém

\section{Metody}
\subsection{Hartree-Fock}
Hartree-Fockova metoda (HF), též nazývaná metoda self-konzistentního pole, je jedna z nejjednodušších ab-initio metod, spočívající v optimalizaci jediného Slaterova determinantu.
