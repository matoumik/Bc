\chapter{Teorie}
\section{}
Základní cíl kvantově chemických výpočtů je najít řešení stacionární schrödingerovy rovnice
\begin{equation}
\op{H}\ket{\Psi} = E\ket{\Psi},
\end{equation}
kde $\ket{\Psi}$ je mnohačásticová vlnová funkce a $\op{H}$ je hamiltonián popisující 
danou molekulu.
Protože se jedná o dost složitý problém na numerické výpočty, je nutné zavést 
několik zjednodušení. 

První je Born-Oppenhaimerova aproximace, která vzhledem k řádově 
rozdílným hmotnostem jader a elektronů rozděluje pohyb jader a elektronů, čímž pádem 
$\ket{\Psi}$ závisí na souřadnicích jader jen parametricky, 
a nevystupují jako proměnné v 
řešené rovnici. 

Další snahou je popis mnohaelektronové vlnové funkce
$\ket{\Psi}:= f(\vec{x}_1 \dots \vec{x}_N)$, kde $\vec{x}_i$ jsou 
polohy jednotlivých elektronů, pomocí součinu jednoelektronových funkcí
$\ket{\Psi}:= f_1(\vec{x}_1)f_2(\vec{x}_2)\dots f_N(\vec{x}_N)$.
Pak ale narážíme na požadavek antisymetrie vlnové funkce vůči prohození libovolných 2 
elektronů. Proto používáme vlnové funkce ve tvaru Slaterova determinantu
\begin{equation}
\ket{\Psi} = \frac{1}{\sqrt{N!}}\begin{vmatrix}
f_1(\vec{x}_1) & f_2(\vec{x}_2)& \cdots & f_N(\vec{x}_1) \\
f_1(\vec{x}_2) & f_2(\vec{x}_2)&        & f_N(\vec{x}_2)\\
\vdots         &               & \ddots & \vdots \\
f_1(\vec{x}_N) & f_2(\vec{x}_N)& \cdots & f_N(\vec{x}_N)
\end{vmatrix}
\end{equation}
Dále musíme tuto funkci rozvinout do nějaké báze. Úplná báze prostoru, na kterém 
pracujeme, by byla nekonečná. Proto musíme najít takovou bázi, abychom mohli problém 
řešit s dostatečnou přesností v konečné bázi.\footnote{I když pořád platí: \uv{Čím víc, 
tím líp.}}
\section{Metody}
\subsection{Báze}
Jako základ pro konstrukci řešení se v kvantové chemii berou orbitaly vodíku podobných atomů, se středem na jednotlivých jádrech. Jejich lineární kombinací získáváme molekulové orbitaly, kde optimalizací koeficientů této lineární kombinace se snažíme dosáhnout toho, aby získané orbitaly byly vlastními stavy hamiltoniánu dané molekuly.
Prvním krokem ovšem je napočítat skalární součiny mezi jednotlivými bázovými vektory a matici hamiltoniánu v dané bázi. To se ukázalo být výpočetně náročné, neboť je třeba numericky řešit velké množství integrálů. Proto se radiální část aproximuje lineární kombinací několika gausovských funkcí, kde již velká část integrálů má analytické vyjádření, čímž se řádově sníží čas výpočtu.
\subsection{Hartree-Fock}
Hartree-Fockova metoda (HF), též nazývaná metoda self-konzistentního pole, je jedna z nejjednodušších ab-initio metod, spočívající v optimalizaci jediného Slaterova determinantu. 
