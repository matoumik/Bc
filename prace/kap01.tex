\chapter{Teorie}
\section{Popis problému}
\label{approx}
Základní cíl kvantově chemických výpočtů je najít řešení stacionární Schrödingerovy rovnice
\begin{equation}
\op{H}\ket{\Psi} = E\ket{\Psi},
\end{equation}
kde $\ket{\Psi}$ je mnohačásticová vlnová funkce a $\op{H}$ je hamiltonián popisující 
danou molekulu, který při zanedbání magnetických i relativistických efektů nabývá tvaru
\begin{equation}
\op{H} =-\sum_{j}\frac{1}{2M_j}\Delta_j 
-\sum_{e}\frac{1}{2m_e}\Delta_e
 -\sum_{j,e} \frac{Z_{j}}{R_{j\,e}}
 +\sum_{j_1,j_2} \frac{Z_{j_1}Z_{j_2}}{R_{j_1 j_2}} 
 +\sum_{e_1,e_2} \frac{1}{R_{e_1 e_2}},
\end{equation}
kde sčítání přes $j$ znamená sčítání přes jádra, $e$ přes elektrony a sčítání přes 
dvojici indexů odpovídá sčítání přes všechny dvojice různých částic daného typu. 
$R_{a\,b}$ pak značí $|\vec{x}_a - \vec{x}_b|$, kde $\vec{x}_{a,b}$ jsou polohy 
jednotlivých částic.

Protože se jedná o dost složitý problém na numerické výpočty, je nutné zavést 
několik zjednodušení. 

Prvním je Born-Oppenhaimerova aproximace, která vzhledem k řádově 
rozdílným hmotnostem jader a elektronů rozděluje pohyb jader a elektronů, čímž pádem 
$\ket{\Psi}$ závisí na souřadnicích jader jen parametricky, 
a nevystupují jako proměnné v 
řešené rovnici. 

Další snahou je popis mnohaelektronové vlnové funkce
$\ket{\Psi}:= f(\vec{x}_1 \dots \vec{x}_N)$, kde $\vec{x}_i$ jsou 
polohy jednotlivých elektronů, pomocí součinu jednoelektronových funkcí
$\ket{\Psi}:= \phi_1(\vec{x}_1)\phi_2(\vec{x}_2)\dots \phi_N(\vec{x}_N)$.
Pak ale narážíme na požadavek antisymetrie vlnové funkce vůči prohození libovolných 2 
elektronů. Proto používáme vlnové funkce ve tvaru Slaterova determinantu
\begin{equation}
\ket{\Psi}_{\mathrm{SD}} = \frac{1}{\sqrt{N!}}\begin{vmatrix}
\phi_1(\vec{x}_1) & \phi_2(\vec{x}_2)& \cdots & \phi_N(\vec{x}_1) \\
\phi_1(\vec{x}_2) & \phi_2(\vec{x}_2)&        & \phi_N(\vec{x}_2)\\
\vdots         &               & \ddots & \vdots \\
\phi_1(\vec{x}_N) & \phi_2(\vec{x}_N)& \cdots & \phi_N(\vec{x}_N)
\end{vmatrix}
\end{equation}
Dále musíme tuto funkci rozvinout do nějaké báze. Úplná báze prostoru, na kterém 
pracujeme, by byla nekonečná. Proto musíme najít takovou bázi, která umožňuje problém 
řešit s dostatečně přesně s konečným počtem prvků.\footnote{I když pořád platí: \uv{Čím víc, tím líp.}}
\subsection{Vibrační hladiny}
Pokud u dvouatomové molekuly známe pro každou vzdálenost jader energii celého systému,
nechovají se ani jádra jako klasická částice, ale chovají se jako jedna částice, pohybující se v získaném potenciálu. Je třeba ovšem pracovat v těžišťové soustavě,
čímž této částici je nutné přiřadit redukovanou hmotnost
\begin{equation}
\mu = \frac{M_1M_2}{M_1+M_2},
\end{equation}
kde $M_1, M_2$ jsou hmotnosti jader. Poté řešíme jednočásticovou Schrödingerovu rovnici pro pohyb v daném potenciálu.
\subsection{Molekulová symetrie}
Symetrie zkoumané molekuly se projeví symetrií jejího hamiltoniánu, která určitým 
způsobem promítne do symetrie vlnové funkce.
Molekuly, jimiž se zabýváme, jsou heteronukleární diatomika, které patří do bodové grupy 
symetrie $C_{\infty_h}$. Tato symetrie se projeví možností klasifikovat vlastní stavy 
pomocí symetrii vůči rotaci kolem spojnice jader, což odpovídá přiřazení určité hodnoty 
průmětu momentu hybnosti do této osy. Používá se značení molekulových stavů
\begin{equation}
^{2S+1}\Lambda^{+/-},\;\;\mathrm{kde }\;\Lambda\in\left\lbrace \Sigma, \Pi,\Delta \dots 
\right\rbrace
\end{equation}
a $S$ je celkový spin molekuly. $+/-$ pak značí změnu znaménka při zrcadlení v rovině 
obsahující osu symetrie, kterou má význam uvažovat jen u $\Sigma$ stavu.

Bohužel většina kvantově-chemických programů neumí pracovat s neabelovskými grupami, 
proto je třeba vzít největší abelovskou bodovou grupu do které molekula , kterou v tomto případě je $C_{2h}$, která má 4 irreducibilní reprezentace umožňující klasifikaci stavů, značené $A_1, A_2, B_1, B_2$. Je tedy třeba před výpočtem hledaný stav molekuly vyjádřit v této reprezentaci.
\section{Metody}
\subsection{Báze}
Jako základ pro konstrukci řešení se v kvantové chemii berou orbitaly vodíku podobných 
atomů, se středem na jednotlivých jádrech. Jejich lineární kombinací získáváme 
molekulové orbitaly, kde optimalizací koeficientů této lineární kombinace se snažíme 
dosáhnout toho, aby získané orbitaly byly vlastními stavy hamiltoniánu dané molekuly.
Prvním krokem ovšem je napočítat skalární součiny mezi jednotlivými bázovými vektory a 
matici hamiltoniánu v dané bázi. To se ukázalo být výpočetně náročné, neboť je třeba 
numericky řešit velké množství integrálů. Proto se radiální část aproximuje lineární 
kombinací několika gaussovských funkcí, kde již velká část integrálů má analytické 
vyjádření, čímž se řádově sníží čas výpočtu.

Bylo již vyvinuto nespočetné množství různých bází lišících se v přesnosti, výpočetní 
náročnosti i metodami, pro které jsou primárně určené.
V našich výpočtech jsme používaly báze vyvíjené 
skupinou T. Dunninga\footnote{cc--pVXZ,  $\mathrm{ X \in \{D,T,Q,5\dots\}}$}\cite{Dunning-basis}, 
které jsou určené pro výpočet post-Hartree-Fockovskými metodami a umožňují extrapolaci 
energie na limitu úplné báze.
Pro popis vlnové funkce ve velkých vzdálenostech od jádra se pak přidávají takzvané 
difuzní funkce, čímž pak získáváme augmented báze\footnote{aug--cc--pVXZ}

\subsection{Hartree-Fock}
Hartree-Fockova metoda (HF), též nazývaná metoda self-konzistentního pole (SCF), je jedna z 
nejjednodušších ab-initio metod, spočívající v optimalizaci jediného Slaterova 
determinantu. Dále uvažuje, že každý elektron se pohybuje ve zprůměrovaném poli 
ostatních elektronů, které je stacionární.
Definujeme pak Fockův operátor
\begin{equation}
\begin{split}
\op{F}\ket{\phi_{e_i}} =& -\frac{1}{2m_e}\Delta\ket{\phi_{e_i}}
-\sum_{j} \frac{Z_{j}}{R_{j\,e_i}}\ket{\phi_{e_i}}
+ \sum_{e \neq e_i} \ket{\phi_{e_i}}\braopket{\phi_{e}}{\frac{1}{R_{e\,e_i}}}{\phi_{e}}
+\\
&+ \sum_{e \neq e_i} \delta_{\sigma_e \sigma_{e_i}}\ket{\phi_{e}}\braopket{\phi_{e_i}}{\frac{1}{R_{e\,e_i}}}{\phi_{e}},
\end{split}
\end{equation} 
pomocí nejž pak zavádíme soustavu nazývanou Hartree-Fockovy rovnice
\begin{equation}
\op{F}\ket{\phi_{e_i}} = \epsilon_i \ket{\phi_{e_i}}.\label{Fockeq}
\end{equation}
Metoda postupuje iterativně, kdy pro nějaký počáteční odhad jednoelektronových funkcí vytvoříme Fockův operátor, u něj najdeme vlastní stavy, které pak použijeme pro vytvoření Fockova operátoru do další iterace.
Řešení pak získáme, pokud funkce získané řešením \eqref{Fockeq} bude rovnat těm použitým ke konstrukci daného Fockova operátoru.
Molekulové orbitaly vystupující v Slaterově determinantu hledáme jako lineární
kombinaci orbitalů 
atomových, kde variací koeficientů této kombinace hledáme řešení problému.

Zatím jsme zanedbávali efekt elektronového spinu. Ten je nutný uvažovat především v 
případě, že není obsazená valenční vrstva. Pak je nutné mít jednu sadu orbitalů pro 
každou projekci spinu zvlášť. Získáme pak metody nazývané restricted-Hartree-Fock 
(RHF), která uvažuje stejný tvar orbitalů pro obě projekce stejné a unrestricted-
Hartree-Fock (UHF), která je má obecně různé pro každou projekci.

\subsection{Multikonfigurační-SCF}
Pro mnoho molekul je popis pomocí jednoho Slaterova determinantu nedostačující.
Jedna z možností, jak toto řešit, je hledat řešení jako lineární kombinaci několika determinantů
\begin{equation}
\ket{\Psi} := \sum_i C_i \ket{\Psi_i}_{\mathrm{SD}},
\end{equation}
kde optimalizujeme zároveň jednotlivé determinanty spolu s koeficienty rozvoje $C_i$, 
podobným způsobem jako u HF metody.
Obesně se tato metoda nazývá multikonfigurační self-konzistentní pole (MC-SCF).
Tato metoda vychází z molekulových orbitalů získaných z HF metody, které používá jako 
prvotní odhad řešení pro optimalizaci.
Pokud jako vstupní odhad použijeme všechny možné Slaterovy determinanty, které je možné  
vytvořit z určité množiny molekulových orbitalů, nazývá se tato metoda complete active space-SCF (CAS-SCF). 

\subsection{Konfigurační interakce}
Metoda Konfigurační interakce (CI) vychází z toho, že molekulové orbitaly získané z Hartree-Fockovy metody tvoří úplnou bázi daného prostoru. V této bázi se pak snažíme hledat řešení ve tvaru
\begin{equation}
\ket{\Psi} := \sum_i C_i \ket{\Psi_i}_{\mathrm{SD}},
\end{equation}
Kde variujeme jen koeficienty rozvoje $C_i$.

Pokud použijeme všechny možné determinanty, které je možné získat v daném prostoru, 
získáváme metodu Full-CI (FCI). Touto metodou získáváme nejpřesnější řešení, za 
platnosti aproximací vyslovených v části \ref{approx} pro danou bázi. Tato metoda je 
ovšem výpočetně náročná.

Pokud použijeme funkci získanou metodou MC-SCF a vytvoříme z nich o funkce získané 
excitací elektronů do vyšších orbitalů, které používáme v rozvoji, získáme metodu
multireferenční CI (MRCI). Je třeba ovšem specifikovat aktivní prostor, což je množina
všech orbitalů, do kterých excitujeme. Určujeme jej zadáním počtu orbitalů pro 
jednotlivé symetrie. 

\subsection{Metoda spřažených clusterů}
Nevýhodou metody CI je, že není takzvaně size-konzistentní\footnote{Výjimku v tomto 
ohledu tvoří FCI, ale ta se nedá pro větší systémy použít}. To znamená, že energie 
získaná touto metodou, pokud se použije na systém se dvěma neinteragujícími podsystémy 
není rovna součtu energií z každého podsystému získaných tou samou metodou. 

Proto existuje metoda spřažených clusterů (CC), která zavádí excitační operátor
\begin{equation}
\op{T} = \op{T}_1 + \op{T}_2 + \op{T}_3 + \dots,
\end{equation}
kde $\op{T}_i$ je lineární kombinace všech takových operátorů, které excitují $i$-
elektronů z nějakého obsazeného orbitalu do nějakého neobsazeného.
Koeficienty této lineární kombinace předem neznáme a řešení hledáme právě
jejich optimalizací.
Vlnovou funkci pak hledáme ve tvaru
\begin{equation}
\ket{\Psi} := \exp(\op{T}) \ket{\Psi_\mathrm{HF}}
\end{equation}
kde $\ket{\Psi_\mathrm{HF}}$ je základní stav získaný Hartree-Fockovou metodou.
Bohužel použít úplný excitační operátor je příliš náročné,\footnote{Přece jen je to 
ekvivalentní FCI} používají se jen první dva členy, čímž získáme metodu CCSD, případně 
případně třetí započítáme perturbativně, čímž získáme metodu CCSD(T).

\subsection{Kvantová teorie na mříži}
Kvantová teorie na mříži je velmi podobná metodě konečných prvků.
Vlnovou funkci vyjádříme pomocí jejích hodnot v ekvidistantně rozmístěných bodech.
Druhou derivaci aproximujeme pomocí hodnot v daných bodech
\begin{equation}
\frac{\partial^2 \Psi}{\partial x^2}(x_i) \approx \frac{2\Psi(x_i)-\Psi(x_{i-1})-
\Psi(x_{i+1})}{\Delta x^2},
\end{equation}
kde $\Delta x$ je rozestup bodů.
S touto aproximací můžeme přepsat Schrödingerovu rovnici jako maticový problém
\begin{equation}
-\frac{1}{2m\Delta x^2} \Psi(x_{i-1}) 
+ \left(\frac{1}{m\Delta x^2}+V(x_i)\right) \Psi(x_i)
-\frac{1}{2m\Delta x^2} \Psi(x_{i+1}) = E\,\Psi(x_i),
\end{equation}
což je standardní problém vlastních čísel tridiagonální matice.
Vlastní vektory jsou pak hodnoty vlnové funkce v daných bodech.
