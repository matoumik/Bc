\chapter{Teorie}
V teoretickém popisu metod budou použity přirozené jednotky, pro které platí
$$ \hbar = m_e = \frac{1}{4\pi\epsilon_0} = 1. $$
\section{Popis problému}
\label{approx}
V této práci se zabýváme elektronovými a vibračními stavy vybraných molekul. Stacionární stavy elektronů a jader jsou dány řešením
bezčasové Schrödingerovy rovnice
\begin{equation}
\op{H}\ket{\Psi} = E\ket{\Psi},
\end{equation}
kde $\ket{\Psi}$ je mnohačásticová vlnová funkce a $\op{H}$ je hamiltonián popisující 
danou molekulu, který pro dvouatomovou molekulu při zanedbání relativistických efektů
nabývá tvaru
\begin{equation}
\op{H} =-\sum_{I}\frac{1}{2M_I}\Delta_I 
-\sum_{i}\frac{1}{2}\Delta_i
 -\sum_{I,i} \frac{Z_{I}}{|\vec{X}_I - \vec{x}_i|}
 +\sum_{I,J} \frac{Z_{J}Z_{I}}{|\vec{X}_I - \vec{X}_J|} 
 +\sum_{i,j:i<j} \frac{1}{|\vec{x}_i - \vec{x}_j|},
\end{equation}
kde velká písmena u veličin i indexů odpovídají jádrům, malé pak jednotlivým 
elektronům a $\vec{x}$ jsou polohy.

Protože se jedná o dost složitý problém i pro numerické výpočty, je nutné zavést 
několik zjednodušení. 

Prvním je Bornova-Oppenheimerova aproximace, která vzhledem k řádově 
rozdílným hmotnostem popisuje odděleně pohyb jader a elektronů, čímž 
pádem umožňuje vlnovou funkci systému zapsat ve formě $\ket{\Psi} \approx 
\ket{\psi_{jad}}\ket{\psi_{el}}$, přičemž $\ket{\psi_{el}}$ závisí na souřadnicích 
jader jen parametricky a jaderný pohyb je popsán čistě pomocí $\ket{\psi_{jad}}$.
To nám zjednodušuje řešení problému, protože můžeme hledat pro pevné rozmístění jader 
$\ket{\psi_{el}}$ a v hamiltoniánu pak nemusíme uvažovat členy závisející jen na 
polohách jader.

Vlnové funkce popisující stavy elektronů uvažujeme ve tvaru Slaterova determinantu
\begin{equation}
\braket{\vec{x}}{\Psi}_{\mathrm{SD}} = \frac{1}{\sqrt{N!}}\begin{vmatrix}
\phi_1(\vec{x}_1) & \phi_2(\vec{x}_2)& \cdots & \phi_N(\vec{x}_1) \\
\phi_1(\vec{x}_2) & \phi_2(\vec{x}_2)&        & \phi_N(\vec{x}_2)\\
\vdots         &               & \ddots & \vdots \\
\phi_1(\vec{x}_N) & \phi_2(\vec{x}_N)& \cdots & \phi_N(\vec{x}_N),
\end{vmatrix}
\end{equation}
a jejich lineární kombinací. To vzhledem k matematickým vlastnostem determinantu zajistí úplnou antisymetrii  vzhledem k prohození libovolných dvou elektronů, která je vyžadována Pauliho vylučovacím principem. $\phi_i$ pak značí jednoelektronové funkce, které též nazýváme molekulové orbitaly.

\subsection{Molekulová symetrie}
Symetrie zkoumané molekuly se projeví invariancí jejího hamiltoniánu vůči některým operacím, která se pak určitým 
způsobem promítne do symetrie vlnové funkce.
Molekuly, jimiž se zabýváme, jsou heteronukleární diatomika, které patří do bodové grupy 
symetrie $C_{\infty v}$. Tato symetrie se projeví možností klasifikovat vlastní stavy 
pomocí ireducibilních reprezentací této grupy, které odpovídají určitým hodnotám 
průmětu celkového orbitálního momentu hybnosti do této osy. Používá se značení molekulových stavů
\begin{equation}
^{2S+1}\Lambda^{+/-},\;\;\mathrm{kde }\;\Lambda\in\left\lbrace \Sigma, \Pi,\Delta \dots 
\right\rbrace
\end{equation}
a $S$ je celkový spin molekuly. $+/-$ pak značí změnu znaménka při zrcadlení v rovině 
obsahující osu symetrie, kterou má význam uvažovat jen u $\Sigma$ stavů.

Bohužel většina kvantově-chemických programů neumí pracovat s neabelovskými grupami, 
proto je třeba vzít největší abelovskou bodovou grupu do které molekula náleží. V 
tomto 
případě je to $C_{2v}$, která má 4 ireducibilní reprezentace umožňující klasifikaci 
stavů, značené $A_1, A_2, B_1, B_2$. Při výpočtech pro určitý stav je tedy třeba 
předem zadat, do které z těchto reprezentací hledaný stav patří.
\section{Metody}
\subsection{Hartreeho-Fockova metoda}
Hartreeho-Fockova metoda (HF), též nazývaná metoda self-konzistentního pole (SCF), je 
jedna z 
nejjednodušších ab-initio metod, spočívající v optimalizaci jednoelektronových 
vlnových funkcí $\phi_{i}$, takzvaných molekulových orbitalů, tvořících jediný Slaterův 
determinant. Dále uvažuje, že každý elektron se pohybuje ve zprůměrovaném poli 
ostatních elektronů, které je stacionární.
Definujeme Fockův operátor
\begin{equation}
\begin{split}
\op{F}\ket{\phi_{i}} =& -\frac{1}{2}\Delta\ket{\phi_{i}}
-\sum_{I} \frac{Z_{I}}{|\vec{X}_I - \vec{x}_i|}\ket{\phi_{i}}
+ \sum_{j \neq i} \ket{\phi_{i}}\braopket{\phi_{j}}{\frac{1}{|\vec{x}_i - \vec{x}_j|}}{\phi_{j}}
+\\
&+ \sum_{j \neq i} \delta_{\sigma_i \sigma_{j}}\ket{\phi_{j}}\braopket{\phi_{i}}{\frac{1}{|\vec{x}_i - \vec{x}_j|}}{\phi_{j}},
\end{split}
\end{equation} 
a hledané molekulové orbitaly jsou řešením soustavy nazývané Hartreeho-Fockovy rovnice
\begin{equation}
\op{F}\ket{\phi_{e_i}} = \epsilon_i \ket{\phi_{e_i}}.\label{Fockeq}
\end{equation}
Metoda postupuje iterativně, kdy pro nějaký počáteční odhad jednoelektronových funkcí 
vytvoříme Fockův operátor, u něj najdeme vlastní stavy, které pak použijeme pro 
vytvoření Fockova operátoru do další iterace.
Řešení pak získáme, pokud funkce získané řešením \eqref{Fockeq} budou dostatečně 
blízké 
těm použitým ke konstrukci příslušného Fockova operátoru.

Molekulové orbitaly vystupující v Slaterově determinantu nejčastěji hledáme jako 
lineární
kombinaci bázových funkcí, kde variací koeficientů této kombinace hledáme řešení 
problému.
To umožňuje přepsat Hartree-Fockovy rovnice \eqref{Fockeq} jako maticový problém, též 
nazývaný Roothanovy rovnice.
Metoda je variační, a tedy energie získaná touto metodou je nutně vyšší než 
správné vlastní stavy zkoumaného hamiltoniánu.

Zatím jsme nediskutovali efekt elektronového spinu. Ten je nutný uvažovat především v 
případě, že není plně obsazená valenční vrstva. Pak je nutné mít jednu sadu orbitalů pro 
každou projekci spinu zvlášť. Získáme 
pak dvě různé metody nazývané restricted-Hartree-Fock 
(RHF), která uvažuje stejný tvar orbitalů pro obě projekce stejné, a unrestricted-
Hartree-Fock (UHF), která je má obecně různé pro každou projekci.


\subsection{Multikonfigurační-SCF}
Pro mnoho molekul je popis pomocí jednoho Slaterova determinantu nedostačující.
Jedna z možností, jak toto řešit, je hledat řešení jako lineární kombinaci několika determinantů
\begin{equation}
\ket{\Psi} = \sum_i C_i \ket{\Psi_i}_{\mathrm{SD}}, \label{MCSCF}
\end{equation}
kde optimalizujeme zároveň jednotlivé molekulové orbitaly spolu s koeficienty rozvoje $C_i$, 
podobným způsobem jako u HF metody.
Obecně se tato metoda nazývá multikonfigurační self-konzistentní pole (MC-SCF).
Tato metoda často vychází z molekulových orbitalů získaných z HF metody, které používá jako 
prvotní odhad řešení pro optimalizaci.
Pokud v rozvoji \eqref{MCSCF} použijeme všechny Slaterovy determinanty, které je možné  
vytvořit z určité množiny molekulových orbitalů, nazývá se tato metoda complete active 
space-SCF (CAS-SCF). 

\subsection{Konfigurační interakce}
Metoda konfigurační interakce (CI) uvažuje řešení ve tvaru
\begin{equation}
\ket{\Psi} = \sum_i C_i \ket{\Psi_i}_{\mathrm{SD}}, \label{CI}
\end{equation}
kde variujeme jen koeficienty rozvoje $C_i$, přičemž použité determinanty jsou 
vytvořené kombinováním molekulových orbitalů získaných jinou metodou. 

Pokud použijeme všechny determinanty, které je možné získat v daném prostoru, 
získáváme metodu Full-CI (FCI). Touto metodou získáváme nejpřesnější řešení, za 
platnosti aproximací vyslovených v části \ref{approx}, pro danou bázi. Tato metoda je 
ovšem výpočetně náročná.

Pokud použijeme funkci získanou metodou MC-SCF a uděláme rozvoj \ref{CI} s použitím všech Slaterových determinantů získaných 
excitací jednoho nebo dvou elektronů do některého z vyšších orbitalů,
získáme metodu
multireferenční CI (MRCI). Tyto metody, stejně jako MC-SCF, jsou variační.

\subsection{Metoda spřažených clusterů}
Nevýhodou metody CI je, že není takzvaně size-konzistentní\footnote{Výjimku v tomto 
ohledu tvoří FCI, ale ta se nedá pro větší systémy použít}. To znamená, že energie 
získaná touto metodou, pokud se použije na systém se dvěma neinteragujícími podsystémy,
není rovna součtu energií z každého podsystému získaných tou samou metodou. 

Proto byla navržena metoda spřažených clusterů (CC), která zavádí excitační operátor
\begin{equation}
\op{T} = \op{T}_1 + \op{T}_2 + \op{T}_3 + \dots,
\end{equation}
kde $\op{T}_i$ je lineární kombinace všech takových operátorů, které excitují $i$ elektronů z obsazených orbitalů do neobsazených.
Koeficienty této lineární kombinace předem neznáme a řešení hledáme právě
jejich optimalizací.
Vlnovou funkci pak hledáme ve tvaru
\begin{equation}
\ket{\Psi} = \exp(\op{T}) \ket{\Psi_\mathrm{HF}}
\end{equation}
kde $\ket{\Psi_\mathrm{HF}}$ je základní stav získaný Hartreeho-Fockovou metodou.
Protože použít úplný excitační operátor je příliš náročné -- je to 
ekvivalentní FCI, používají se jen první dva členy, čímž získáme metodu CCSD, 
případně třetí započítáme perturbativně, čímž získáme metodu CCSD(T). 
Tato metoda ovšem už není variační, takže může dát i nižší energii než je přesná hodnota. 

\subsection{Báze}
Řešení je třeba vyjádřit v nějaké bázi, 
Při řešení je nutné počítat skalární součiny mezi jednotlivými bázovými funkcemi a vyjádřit
hamiltonián, případně použité operátory Fockův, excitační v CCSD) v dané bázi. To se ukazuje být výpočetně náročné, neboť je třeba 
numericky řešit velké množství integrálů. Proto se používají gaussovské báze, kde 
již velká část integrálů má analytické 
vyjádření, čímž se řádově snižuje doba výpočtu.

Bylo již navrženo velké množství různých bází lišících se přesností, výpočetní 
náročností i metodami, pro které jsou primárně určené.
V našich výpočtech jsme používaly báze cc--pVXZ,  $\mathrm{ X \in \{D,T,Q,5\dots\}}$ vyvíjené 
skupinou T. Dunninga \cite{Dunning-basis}, 
které jsou určené pro výpočet post-Hartree-Fockovskými metodami a umožňují extrapolaci 
energie na limitu úplné báze.
Pro popis vlnové funkce ve velkých vzdálenostech od jádra se pak přidávají takzvané 
difuzní funkce, čímž pak získáváme takzvané augmented báze aug--cc--pVXZ


\subsection{Vibrační hladiny}
\label{te_vibr}
Pokud u dvouatomové molekuly známe pro každou vzdálenost jader energii celého systému, 
lze přejít do těžišťové soustavy a řešit Schrödingerovu rovnici pro pohyb jader,
přičemž hamiltonián pro vibrační pohyb jader má tvar
\begin{equation}
\frac{1}{2\mu}\frac{\partial^2}{\partial R^2} + V(R),
\end{equation}
kde $V(R)$ je energie získaná řešením elektronového problému a $\mu$ je redukovaná 
hmotnost, definovaná
\begin{equation}
\mu = \frac{M_1M_2}{M_1+M_2},
\end{equation}
kde $M_1, M_2$ jsou hmotnosti jader. 
Nalezením vlastních stavů hamiltoniánu získáme pak vibrační hladiny daného systému.

Tento problém řešíme numericky, k čemuž používáme Numerovovu metodu.
Vlnovou funkci i potenciál vyjádříme jako pomocí jejich hodnot v ekvidistantně rozmístěných 
bodech.
Zapíšeme vyjádření Schrodingerovy rovnice v Numerově metodě pomocí maticového zápisu
\begin{equation}
(-\frac{1}{2\mu}\mathbb{A} + \mathbb{B}\dot(\mathbb{V} - E\mathbb{I})\Psi = 0
\end{equation}
kde $\mathbb{I}$ je jednotková matice, $\mathbb{V}$ je matice s hodnotami potenciálu v daných bodech na diagonále a 
matice $\mathbb{A},\mathbb{B}$ vypadají
$$ \mathbb{A} = \frac{1}{\Delta x^2}\begin{pmatrix}
-2 & 1 & 0 & \cdots & 0 \\  
1 & -2 & 1 & \ddots & \vdots \\ 
0 & 1 & \ddots & \ddots & 0 \\  
\vdots & \ddots & \ddots & \ddots & 1 \\ 
0 & \cdots & 0 & 1 & -2 \\ 
\end{pmatrix}
$$
$$
\mathbb{B} = 
\frac{1}{12}
\begin{pmatrix}
10 & 1 & 0 & \cdots & 0 \\ 
1 & 10 & 1 & \ddots & \vdots \\ 
0 & 1 & \ddots & \ddots & 0 \\ 
\vdots & \ddots & \ddots & \ddots & 1 \\  
0 & \cdots & 0 & 1 & 10 \\ 
\end{pmatrix}
$$
přičemž $\Delta x$ je rozestup bodů. Pak jednoduchou úpravou získáme
\begin{equation}
\left(-\frac{1}{2\mu}\mathbb{A}\mathbb{B}^{-1}+\mathbb{V}\right)\vec{\Psi} = E 
\vec{\Psi}
\end{equation}
což je standardní problém vlastních čísel matice.
Jeko řešením získáme vlastní čísla, která jsou rovna energiím vlastních stavů systému 
a vlastní vektory, které jsou hodnoty vlnové funkce v daných bodech.
