\chapter{Teorie}
\section{Popis problému}
\label{approx}
stacionární Schrödingerovy rovnice
\begin{equation}
\op{H}\ket{\Psi} = E\ket{\Psi},
\end{equation}
kde $\ket{\Psi}$ je mnohačásticová vlnová funkce a $\op{H}$ je hamiltonián popisující 
danou molekulu, který pro dvouatomovou molekulu při zanedbání magnetických i relativistických efektů nabývá tvaru
\begin{equation}
\op{H} =-\sum_{I}\frac{1}{2M_I}\Delta_I 
-\sum_{i}\frac{1}{2m_i}\Delta_i
 -\sum_{I,i} \frac{Z_{I}}{|\vec{X}_I - \vec{x}_i|}
 +\sum_{I,J} \frac{Z_{J}Z_{I}}{|\vec{X}_I - \vec{X}_J|} 
 +\sum_{i,j:i<j} \frac{1}{|\vec{x}_i - \vec{x}_j|},
\end{equation}
kde velká písmena u veličin i indexů odpovídají jádrům, malé pak jednotlivým 
elektronům. kde $\vec{x}, \vec{X}$ jsou polohy 
dané částice.

Protože se jedná o dost složitý problém i pro numerické výpočty, je nutné zavést 
několik zjednodušení. 

Prvním je Bornova-Oppenhaimerova aproximace, která vzhledem k řádově 
rozdílným hmotnostem jader a elektronů popisuje odděleně pohyb jader a elektronů, čímž 
pádem umožňuje vlnovou funkci systému zapsat ve formě $\ket{\Psi} \approx 
\ket{\psi_{jad}}\ket{\psi_{el}}$, přičemž $\ket{\psi_{el}}$ závisí na souřadnicích 
jader jen parametricky a jaderný pohyb je popsán čistě pomocí $\ket{\psi_{jad}}$.
To nám zjednodušuje řešení problému, protože můžeme hledat pro pevné rozmístění jader 
$\ket{\psi_{el}}$ a v hamiltoniánu pak nemusíme uvažovat členy závisející jen na 
polohách jader.

Dále používáme vlnové funkce ve tvaru Slaterova determinantu
\begin{equation}
\braket{\vec{x}}{\Psi}_{\mathrm{SD}} = \frac{1}{\sqrt{N!}}\begin{vmatrix}
\phi_1(\vec{x}_1) & \phi_2(\vec{x}_2)& \cdots & \phi_N(\vec{x}_1) \\
\phi_1(\vec{x}_2) & \phi_2(\vec{x}_2)&        & \phi_N(\vec{x}_2)\\
\vdots         &               & \ddots & \vdots \\
\phi_1(\vec{x}_N) & \phi_2(\vec{x}_N)& \cdots & \phi_N(\vec{x}_N),
\end{vmatrix}
\end{equation}
případně jejich lineární kombinace. To vzhledem k matematickým vlastnostem determinantu zajistí úplnou antisymetrii  vzhledem k prohození libovolných dvou elektronů, která je vyžadována Pauliho vylučovacím principem. $\phi_i$ pak značí jednoelektronové funkce, které též nazýváme molekulové orbitaly.

\subsection{Molekulová symetrie}
Symetrie zkoumané molekuly se projeví invariancí jejího hamiltoniánu vůči některým operacím, která se pak určitým 
způsobem promítne do symetrie vlnové funkce.
Molekuly, jimiž se zabýváme, jsou heteronukleární diatomika, které patří do bodové grupy 
symetrie $C_{\infty_v}$. Tato symetrie se projeví možností klasifikovat vlastní stavy 
pomocí ireducibilních reprezentací této grupy, které odpovídají určitým hodnotám 
průmětu momentu hybnosti do této osy. Používá se značení molekulových stavů
\begin{equation}
^{2S+1}\Lambda^{+/-},\;\;\mathrm{kde }\;\Lambda\in\left\lbrace \Sigma, \Pi,\Delta \dots 
\right\rbrace
\end{equation}
a $S$ je celkový spin molekuly. $+/-$ pak značí změnu znaménka při zrcadlení v rovině 
obsahující osu symetrie, kterou má význam uvažovat jen u $\Sigma$ stavu.

Bohužel většina kvantově-chemických programů neumí pracovat s neabelovskými grupami, 
proto je třeba vzít největší abelovskou bodovou grupu do které molekula náleží. V tomto 
případě je to $C_{2v}$, která má 4 ireducibilní reprezentace umožňující klasifikaci 
stavů, značené $A_1, A_2, B_1, B_2$. Je tedy třeba před výpočtem hledaný stav molekuly 
vyjádřit v této reprezentaci.
\section{Metody}
\subsection{Hartreeho-Fockova metoda}
Hartreeho-Fockova metoda (HF), též nazývaná metoda self-konzistentního pole (SCF), je jedna z 
nejjednodušších ab-initio metod, spočívající v optimalizaci jediného Slaterova 
determinantu. Dále uvažuje, že každý elektron se pohybuje ve zprůměrovaném poli 
ostatních elektronů, které je stacionární.
Definujeme pak Fockův operátor
\begin{equation}
\begin{split}
\op{F}\ket{\phi_{i}} =& -\frac{1}{2m_e}\Delta\ket{\phi_{i}}
-\sum_{I} \frac{Z_{I}}{|\vec{X}_I - \vec{x}_i|}\ket{\phi_{i}}
+ \sum_{j \neq i} \ket{\phi_{i}}\braopket{\phi_{j}}{\frac{1}{|\vec{x}_i - \vec{x}_j|}}{\phi_{j}}
+\\
&+ \sum_{j \neq i} \delta_{\sigma_i \sigma_{j}}\ket{\phi_{j}}\braopket{\phi_{i}}{\frac{1}{|\vec{x}_i - \vec{x}_j|}}{\phi_{j}},
\end{split}
\end{equation} 
pomocí nějž pak zavádíme soustavu nazývanou Hartreeho-Fockovy rovnice
\begin{equation}
\op{F}\ket{\phi_{e_i}} = \epsilon_i \ket{\phi_{e_i}}.\label{Fockeq}
\end{equation}
Metoda postupuje iterativně, kdy pro nějaký počáteční odhad jednoelektronových funkcí 
vytvoříme Fockův operátor, u něj najdeme vlastní stavy, které pak použijeme pro 
vytvoření Fockova operátoru do další iterace.
Řešení pak získáme, pokud funkce získané řešením \eqref{Fockeq} budou dostatečně blízké 
těm použitým ke konstrukci příslušného Fockova operátoru.

Molekulové orbitaly vystupující v Slaterově determinantu nejčastěji hledáme jako lineární
kombinaci bázových funkcí, kde variací koeficientů této kombinace hledáme řešení problému.
To umožňuje přepsat Hartree-Fockovy rovnice \eqref{Fockeq} jako maticový problém, též nazývané Roothanovy rovnice.
Metoda je variační, a tedy energie získaná touto metodou je nutně vyšší než 
správné vlastní stavy zkoumaného hamiltoniánu.

Zatím jsme zanedbávali efekt elektronového spinu. Ten je nutný uvažovat především v 
případě, že není obsazená valenční vrstva. Pak je nutné mít jednu sadu orbitalů pro 
každou projekci spinu zvlášť. Získáme pak dvě různé metody nazývané restricted-Hartree-Fock 
(RHF), která uvažuje stejný tvar orbitalů pro obě projekce stejné a unrestricted-
Hartree-Fock (UHF), která je má obecně různé pro každou projekci.

\subsection{Báze}
Při řešení je nutné počítat skalární součiny mezi jednotlivými bázovými funkcemi a vyjádřit
hamiltonián\footnote{Respektive použité operátory (Fockův, \dots)} v dané bázi. To se ukazuje být výpočetně náročné, neboť je třeba 
numericky řešit velké množství integrálů. Proto se se používají gaussovské báze, kde 
již velká část integrálů má analytické 
vyjádření, čímž se řádově snižuje doba výpočtu.

Bylo již vyvinuto nespočetné množství různých bází lišících se v přesnosti, výpočetní 
náročnosti i metodami, pro které jsou primárně určené.
V našich výpočtech jsme používaly báze vyvíjené 
skupinou T. Dunninga\footnote{cc--pVXZ,  $\mathrm{ X \in \{D,T,Q,5\dots\}}$}\cite{Dunning-basis}, 
které jsou určené pro výpočet post-Hartree-Fockovskými metodami a umožňují extrapolaci 
energie na limitu úplné báze.
Pro popis vlnové funkce ve velkých vzdálenostech od jádra se pak přidávají takzvané 
difuzní funkce, čímž pak získáváme augmented báze\footnote{aug--cc--pVXZ}

\subsection{Multikonfigurační-SCF}
Pro mnoho molekul je popis pomocí jednoho Slaterova determinantu nedostačující.
Jedna z možností, jak toto řešit, je hledat řešení jako lineární kombinaci několika determinantů
\begin{equation}
\ket{\Psi} = \sum_i C_i \ket{\Psi_i}_{\mathrm{SD}},
\end{equation}
kde optimalizujeme zároveň jednotlivé determinanty spolu s koeficienty rozvoje $C_i$, 
podobným způsobem jako u HF metody.
Obesně se tato metoda nazývá multikonfigurační self-konzistentní pole (MC-SCF).
Tato metoda vychází z molekulových orbitalů získaných z HF metody, které používá jako 
prvotní odhad řešení pro optimalizaci.
Pokud jako vstupní odhad použijeme všechny možné Slaterovy determinanty, které je možné  
vytvořit z určité množiny molekulových orbitalů, nazývá se tato metoda complete active space-SCF (CAS-SCF). 

\subsection{Konfigurační interakce}
Metoda Konfigurační interakce (CI) vychází z toho, že molekulové orbitaly získané z Hartree-Fockovy metody tvoří úplnou bázi daného prostoru. V této bázi se pak snažíme hledat řešení ve tvaru
\begin{equation}
\ket{\Psi} = \sum_i C_i \ket{\Psi_i}_{\mathrm{SD}},
\end{equation}
Kde variujeme jen koeficienty rozvoje $C_i$.

Pokud použijeme všechny možné determinanty, které je možné získat v daném prostoru, 
získáváme metodu Full-CI (FCI). Touto metodou získáváme nejpřesnější řešení, za 
platnosti aproximací vyslovených v části \ref{approx} pro danou bázi. Tato metoda je 
ovšem výpočetně náročná.

Pokud použijeme funkci získanou metodou MC-SCF a vytvoříme z nich o funkce získané 
excitací elektronů do vyšších orbitalů, které používáme v rozvoji, získáme metodu
multireferenční CI (MRCI). Je třeba ovšem specifikovat aktivní prostor, což je množina
všech orbitalů, do kterých excitujeme. Určujeme jej zadáním počtu orbitalů pro 
jednotlivé symetrie. Tyto metody, stejně jako MC-SCF, jsou variační.

\subsection{Metoda spřažených clusterů}
Nevýhodou metody CI je, že není takzvaně size-konzistentní\footnote{Výjimku v tomto 
ohledu tvoří FCI, ale ta se nedá pro větší systémy použít}. To znamená, že energie 
získaná touto metodou, pokud se použije na systém se dvěma neinteragujícími podsystémy 
není rovna součtu energií z každého podsystému získaných tou samou metodou. 

Proto existuje metoda spřažených clusterů (CC), která zavádí excitační operátor
\begin{equation}
\op{T} = \op{T}_1 + \op{T}_2 + \op{T}_3 + \dots,
\end{equation}
kde $\op{T}_i$ je lineární kombinace všech takových operátorů, které excitují $i$-
elektronů z nějakého obsazeného orbitalu do nějakého neobsazeného.
Koeficienty této lineární kombinace předem neznáme a řešení hledáme právě
jejich optimalizací.
Vlnovou funkci pak hledáme ve tvaru
\begin{equation}
\ket{\Psi} = \exp(\op{T}) \ket{\Psi_\mathrm{HF}}
\end{equation}
kde $\ket{\Psi_\mathrm{HF}}$ je základní stav získaný Hartree-Fockovou metodou.
Bohužel použít úplný excitační operátor je příliš náročné,\footnote{Přece jen je to 
ekvivalentní FCI} používají se jen první dva členy, čímž získáme metodu CCSD, případně 
případně třetí započítáme perturbativně, čímž získáme metodu CCSD(T). 
Tyto metody ovšem už nejsou variační, takže mohou dát i nižší energii než je přesná hodnota. 

\subsection{Vibrační hladiny}
\label{te_vibr}
Pokud u dvouatomové molekuly známe pro každou vzdálenost jader energii celého systému, lze přejít do těžišťové soustavy a řešit Schrödingerovu rovnici pro pohyb jader,
přičemž hamiltonián nabývá tvaru\footnote{Rotaci molekuly neuvažujeme.}
\begin{equation}
\frac{1}{2\mu}\frac{\partial^2}{\partial R^2} + V(R),
\end{equation}
kde $V(R)$ je energie získaná řešením elektronového problému a $\mu$ je redukovaná hmotnost, definovaná
\begin{equation}
\mu = \frac{M_1M_2}{M_1+M_2},
\end{equation}
kde $M_1, M_2$ jsou hmotnosti jader. 
Tímto způsobem získáme pak vibrační hladiny daného systému.

Tento problém řešíme numericky, přičemž používáme
metodu, která je velmi podobná metodě konečných diferencí.
Vlnovou funkci vyjádříme pomocí jejích hodnot v ekvidistantně rozmístěných bodech.
Druhou derivaci aproximujeme pomocí hodnot v daných bodech
\begin{equation}
\frac{\partial^2 \Psi}{\partial x^2}(x_i) \approx \frac{\Psi(x_{i-1})-2\Psi(x_i)+
\Psi(x_{i+1})}{\Delta x^2},
\end{equation}
kde $\Delta x$ je rozestup bodů.
S touto aproximací můžeme přepsat Schrödingerovu rovnici jako maticový problém
\begin{equation}
-\frac{1}{2m\Delta x^2} \Psi(x_{i-1}) 
+ \left(\frac{1}{m\Delta x^2}+V(x_i)\right) \Psi(x_i)
-\frac{1}{2m\Delta x^2} \Psi(x_{i+1}) = E\,\Psi(x_i),
\end{equation}
což je standardní problém vlastních čísel tridiagonální matice.
Vlastní vektory jsou pak hodnoty vlnové funkce v daných bodech.
